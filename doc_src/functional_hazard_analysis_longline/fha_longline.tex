
\documentclass[a4paper,10pt]{scrartcl}

\usepackage[utf8]{inputenc}
\usepackage{tikz}
\usepackage{amsmath}
\usepackage{hyperref}

\usepackage{subfig}
\usepackage{graphicx}
\usepackage{pgfplots}
\usepackage{pgfplotstable}
\tikzset{>=stealth}
\usetikzlibrary{patterns}
\usetikzlibrary{pgfplots.statistics}

\setlength{\paperheight}{30cm}
\setlength{\textheight}{24.5cm}
\setlength{\paperwidth}{21cm}
\setlength{\textwidth}{16.8cm}
\setlength{\oddsidemargin}{-0.3cm}
\setlength{\evensidemargin}{-0.3cm}
\setlength{\parindent}{0cm}
\setlength{\topmargin}{-1.9cm}
\setlength{\headsep}{1.3cm}
\setlength{\footskip}{1.5cm}

\title{Functional Hazard Analysis of a Helicopter-based Highline Rescue Using a Longline}

\author{Version 1.0 by Jakob Bludau, Aaron Benkert, and Lukas Irmler}



\begin{document}
\maketitle

\tableofcontents

\section{Introduction}
\label{sec:intro}
A Highline rescue is similar to one of a lead climber, a highly technical rescue. Furthermore, the patient is prone to suspension trauma due to long periods of hanging in a harness. Therefore, time is a priority.
Highlines span gaps, ridges, etc. Often, the anchors are only reachable by rappel/climbing. Thus, even if a terrestrial rescue transports the patient to one of the anchors, the way to a hospital is still long. Furthermore, the mode of transport to the hospital is probably a helicopter due to the terrain.\\

With longer Highlines, the rescue is escalating in complexity. Ropes need to be longer, and direct rappel off the Highline is no longer possible or dangerous (stretch in “static” ropes and webbings is not negligible and leads to large vertical movement during rappel). Furthermore, rescuers need to pull the patient up for a rescue towards one of the anchors due to the Highline sag. \\

These thoughts lead the authors to start a discussion with Heli Austria (namely Gabriel Falkner) on a helicopter-based rescue directly off the Highline. A literature review showed that there is little published experience on this topic.\\

This document contains the thoughts of the authors on a helicopter-based highline rescue. The document shows the stages of procedure in section \ref{sec:proc}, based on the procedure for an injured lead climber. Section \ref{sec:conventional} describes the conventional setup of a Highline. From this, the authors derived a functional hazard analysis of the rescue procedure in section \ref{sec:fha}. Section \ref{sec:fha} also contains measures of mitigation for the most critical risks. Section \ref{sec:modified} describes the resulting Highline setup for a safer training. Finally, section \ref{sec:result} lists the procedure and the personal protective equipment in detail. The overall focus is on safety.

This document is targeted on a rescue with a longline as connection between rescuer and helicopter. Inferring from the contact of climbing ropes with the Highline, the authors suspect the textile longline rope with a diameter of approx. 20mm to have a negligible chance of cutting the Highline. If a thin steel hoist cable as used for rotorcraft-based rescue poses serious threat 


\section{Procedure of lead climber rescue as starting point}
\label{sec:proc}

This document starts with a procedure similar to rescuing a lead climber hanging in a harness from a rope. The load-bearing path defines the phases of this procedure. (The names of the phases are in bold font in the following list)

\begin{itemize}
\item  1st: \textbf{Approach and positioning} of the rotorcraft, including rescuer and rescue equipment above the patient.
\item  2nd: Rappel of rescuer and \textbf{attaching of the patient's harness} to the longline.
\item 3rd: \textbf{Climb of the rotorcraft} to transfer the patient's load off the climbing rope onto the longline.
\item 4th: \textbf{Cutting of the climbing rope} and check that the patient and rescuer are only connected to the rotorcraft.
\item 5th: \textbf{Departure} towards a free direction.
\end{itemize}

From the start of phase 2 until the end of phase 4, the rescue equipment, the harness of the patient, and the climbing rope connect the rotorcraft to the terrain. Nevertheless, the stretch and slack of the climbing rope allow movement in a limited range. \\

During phases 1 and 2, the load-bearing paths of the rescuer and the patient are entirely separate. During phase 3, the patient's load is gradually transferred to the longline. The transfer is complete if the climbing rope is free of load. After the climbing rope cut in phase 4, the rotorcraft carries the load of the rescuer and the patient. 

\section{Conventional highline setup}
\label{sec:conventional}

\section{Functional hazard analysis and risk mitigation for a safe training setting}
\label{sec:fha}

\section{Modified highline setup for training}
\label{sec:modified}

\section{Resulting procedure for training}
\label{sec:result}

\end{document}
