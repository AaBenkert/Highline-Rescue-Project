
\documentclass[a4paper,10pt]{scrartcl}

\usepackage[utf8]{inputenc}
\usepackage{tikz}
\usepackage{amsmath}
\usepackage{hyperref}
\usepackage{makecell}

\usepackage{subfig}
\usepackage{graphicx}
\usepackage{pgfplots}
\usepackage{pgfplotstable}
\tikzset{>=stealth}
\usetikzlibrary{patterns}
\usetikzlibrary{pgfplots.statistics}

\setlength{\paperheight}{30cm}
\setlength{\textheight}{24.5cm}
\setlength{\paperwidth}{21cm}
\setlength{\textwidth}{16.8cm}
\setlength{\oddsidemargin}{-0.3cm}
\setlength{\evensidemargin}{-0.3cm}
\setlength{\parindent}{0cm}
\setlength{\topmargin}{-1.9cm}
\setlength{\headsep}{1.3cm}
\setlength{\footskip}{1.5cm}

\title{Functional Hazard Analysis of a Helicopter-based Highline Rescue Using a Longline}

\author{Version 1.1 by Jakob Bludau, Aaron Benkert, and Lukas Irmler}



\begin{document}
\maketitle

\tableofcontents

\section{Introduction}
\label{sec:intro}
A Highline rescue is similar to one of a lead climber, a highly technical rescue. Furthermore, the patient is prone to suspension trauma due to long periods of hanging in a harness. Therefore, time is a priority.
Highlines span gaps, ridges, etc. Often, the anchors are only reachable by rappel/climbing. Thus, even if a terrestrial rescue transports the patient to one of the anchors, the way to a hospital is still long. Furthermore, the mode of transport to the hospital is probably a helicopter due to the terrain.\\

With longer Highlines, the rescue is escalating in complexity. Ropes need to be longer, and direct rappel off the Highline is no longer possible or dangerous (stretch in “static” ropes and webbings is not negligible and leads to large vertical movement during rappel). Furthermore, rescuers need to pull the patient up for a rescue towards one of the anchors due to the Highline sag. \\

These thoughts lead the authors to start a discussion with Heli Austria (namely Gabriel Falkner) on a helicopter-based rescue directly off the Highline. A literature review showed that there is little published experience on this topic.\\

This document contains the thoughts of the authors on a helicopter-based highline rescue. The document shows the stages of procedure in section \ref{sec:proc}, based on the procedure for an injured lead climber. Section \ref{sec:conventional} describes the conventional setup of a Highline. From this, the authors derived a functional hazard analysis of the rescue procedure in section \ref{sec:fha}. Section \ref{sec:fha} also contains measures of mitigation for the most critical risks. Section \ref{sec:modified} describes the resulting Highline setup for a safer training. Finally, section \ref{sec:result} lists the procedure and the personal protective equipment in detail. The overall focus is on safety.

This document is targeted on a rescue with a longline as connection between rescuer and helicopter. Inferring from the contact of climbing ropes with the Highline, the authors suspect the textile longline rope with a diameter of approx. 20mm to have a negligible chance of cutting the Highline. If a thin steel hoist cable as used for rotorcraft-based rescue poses serious threat 


\section{Procedure of lead climber rescue as starting point}
\label{sec:proc}

This document starts with a procedure similar to rescuing a lead climber hanging in a harness from a rope. The load-bearing path defines the phases of this procedure. (The names of the phases are in bold font in the following list)

\begin{itemize}
\item  1st: \textbf{Approach and positioning} of the rotorcraft, including rescuer and rescue equipment above the patient.
\item  2nd: Rappel of rescuer and \textbf{attaching of the patient's harness} to the longline.
\item 3rd: \textbf{Climb of the rotorcraft} to transfer the patient's load off the climbing rope onto the longline.
\item 4th: \textbf{Cutting of the climbing rope} and check that the patient and rescuer are only connected to the rotorcraft.
\item 5th: \textbf{Departure} towards a free direction.
\end{itemize}

From the start of phase 2 until the end of phase 4, the rescue equipment, the harness of the patient, and the climbing rope connect the rotorcraft to the terrain. Nevertheless, the stretch and slack of the climbing rope allow movement in a limited range. \\

During phases 1 and 2, the load-bearing paths of the rescuer and the patient are entirely separate. During phase 3, the patient's load is gradually transferred to the longline. The transfer is complete if the climbing rope is free of load. After the climbing rope cut in phase 4, the rotorcraft carries the load of the rescuer and the patient. \\

This procedure works for a Highline rescue with a few minor modifications. These modifications result from the functional hazard analysis in chapter \ref{sec:fha}. The following shows the resulting procedure. Italics highlight differences between the procedure and the lead climber rescue procedure. 

\begin{itemize}
\item  1st: \textbf{Approach and positioning} of the rotorcraft, including rescuer and rescue equipment above the patient.
\item  2nd: Rappel of rescuer and \textbf{attaching of the patient's harness} to the longline.
\item 3rd: \textbf{Climb of the rotorcraft} to transfer the patient's load off the \textit{Highline} onto the longline. \textit{Furthermore, the rescuer positions the rotorcraft approx. 1m of horizontal distance to the Highline in order to check if any unwanted connection to the Highline exists}.
\item 4th: \textbf{Cutting of the} \textit{leash} and check that the patient and rescuer are only connected to the rotorcraft.
\item 5th: \textbf{Departure} towards a free direction.
\end{itemize}

\section{Conventional highline setup}
\label{sec:conventional}

Conventional Highlines consist of two anchors, a main line, one backup line, a leash ring, the leash, and the athlete's harness. Suitable points for anchors are trees, bolts, and removable equipment like cams, ice screws, or nuts. Anchors are generally redundant. \\
The main and backup lines are hung between the anchors. Personal preference, in combination with the used material, determines the pretension of the main line. The backup line prevents a fatality in case of a mainline failure due to falling debris or friction. It hangs loosely underneath the main line, often in loops. In case of a main line failure, the backup is an entirely redundant connection to the terrain. Nevertheless, this requires sufficient free horizontal distance underneath the line. \\
The main and backup lines go through a closed metal loop called the leash ring. The leash ring is an attachment point that can move along the line. The athlete connects the harness to the leash ring via a 1.2m long leash. This leash often consists of a climbing rope fed through a tubular webbing for redundancy.

\section{Functional hazard analysis and risk mitigation for a safe training setting}
\label{sec:fha}

Two outstanding dangers are present during all phases of the rescue:
\begin{itemize}
\item The backup line, which is a secondary line that runs parallel to the main line, is a crucial component in the rescue operation. It serves as a safety measure in case the main line fails. However, it's important to note that the backup line can also pose a potential danger of entanglement, as its loops can wrap around arms, legs, or material (on the harness, the longline rig, etc.). This can lead to complex problems that are difficult to solve. In a training environment, the use of a tubular material that both main line and backup go through is essential to eliminate the danger of entanglement.
\item In the event of a main line rupture, the line snaps back towards the anchors. This snapping action is more pronounced in what we refer to as ``high slack settings'', which are situations where there is a significant amount of slack in the line. To reduce the risk of line rupture, the slack of a training line is reduced by increasing the tension in the main line. This, in turn, increases the distance between the main line and the rotorcraft, minimizing the potential for the line to snap back towards the rotorcraft.
\end{itemize} 

The following Functional Hazard Analysis lists the risks of every procedure phase separately. It distinguishes between the functional systems \textit{rotorcraft}, \textit{longline}, \textit{harness}, \textit{leash}, \textit{rescuer}, and \textit{highline}. As the patient might be unconscious,   no functions are associated. 


The risks result from errors/inabilities of the listed systems during the rescue phases. This document lists the expected criticality of the identified risks in all phases of the rescue and proposes mitigation to create a safe training environment. 


The text uses \textit{highline} for the system composed of the two anchors, the main and backup line, and the leash ring as this system fulfills the functions \textbf{load bearing} and \textbf{psotitioning} as a single unit. Nevertheless, the risk lists a breakdown in the description if necessary. 


This document gives the criticality of a risk on three levels \textcolor{green}{low}, \textcolor{orange}{medium}, and \textcolor{red}{high}. The levels encode the complexity that a solution to the problem requires and the potential harm for rescuers, patients, and rotorcraft. Table \ref{tab:crit} gives the different levels.

\begin{table}[!ht]
\centering
\begin{tabular}{ c | c | c }
 criticality & complexity of solution & potential harm \\ 
 \hline
 \hline 
 \textcolor{green}{low} & no actions necessary & none \\ 
 \hline
 \textcolor{orange}{medium} & 1-2 simple steps, visual check, known communication & small bruises on arms or legs \\ 
 \hline
 \textcolor{red}{high} & \makecell{cutting, thinking thinking, \\ spontaneous problem solving/communication} & \makecell{everything more than \\ small bruises on arms or legs} \\  
   
\end{tabular}
\caption{Levels of criticality described by their complexity and potential harm.}
\label{tab:crit}
\end{table}

Table \ref{tab:func} gives the functions of the individual systems:
\begin{table}[!ht]
\centering
\begin{tabular}{ c | c  }
 system & function \\ 
 \hline
 \hline 
 rotorcraft & \makecell{load bearing \\ positioning (with help of rescuer)} \\
 \hline 
 rescuer & \makecell{positioning (self,rotatoric) \\ positioning (self,translatoric) \\ attaching patient to longline \\ cutting connection patient to highline} \\
 \hline
 longline & load bearing \\
 \hline
 highline & \makecell{load bearing \\ positioning} \\
 \hline
 leash & load bearing \\
 \hline
 harness (patient) & load bearing \\ 
\end{tabular}
\caption{Systems and respective functions.}
\label{tab:func}
\end{table}

The analysis does not differentiate between different causes of a failure of the rotorcaft to provide the functions \textbf{load bearing} and \textbf{positioning}. The rotorcaft corresponds to a black box system that can change the position in 3D space and lift a load (load bearing). In case of an emergency the rotorcraft can drop the longline.

Humans perform some of the mitigation that is identified by the analysis. Therefore, the three roles \textcolor{teal}{rescuer}, \textcolor{purple}{patient}, and external \textcolor{blue}{observer} are defined. All humans involved in the training are connected via a shared radio.


\section{Modified highline setup for training}
\label{sec:modified}

\section{Resulting procedure for training}
\label{sec:result}

\end{document}
